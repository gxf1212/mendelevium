% Options for packages loaded elsewhere
\PassOptionsToPackage{unicode}{hyperref}
\PassOptionsToPackage{hyphens}{url}
%
\documentclass[
]{article}
\usepackage{amsmath,amssymb}
\usepackage{iftex}
\usepackage{mhchem}
\ifPDFTeX
  \usepackage[T1]{fontenc}
  \usepackage[utf8]{inputenc}
  \usepackage{textcomp} % provide euro and other symbols
\else % if luatex or xetex
  \usepackage{unicode-math} % this also loads fontspec
  \defaultfontfeatures{Scale=MatchLowercase}
  \defaultfontfeatures[\rmfamily]{Ligatures=TeX,Scale=1}
\fi
\usepackage{lmodern}
\ifPDFTeX\else
  % xetex/luatex font selection
\fi
% Use upquote if available, for straight quotes in verbatim environments
\IfFileExists{upquote.sty}{\usepackage{upquote}}{}
\IfFileExists{microtype.sty}{% use microtype if available
  \usepackage[]{microtype}
  \UseMicrotypeSet[protrusion]{basicmath} % disable protrusion for tt fonts
}{}
\makeatletter
\@ifundefined{KOMAClassName}{% if non-KOMA class
  \IfFileExists{parskip.sty}{%
    \usepackage{parskip}
  }{% else
    \setlength{\parindent}{0pt}
    \setlength{\parskip}{6pt plus 2pt minus 1pt}}
}{% if KOMA class
  \KOMAoptions{parskip=half}}
\makeatother
\usepackage{xcolor}
\usepackage{color}
\usepackage{fancyvrb}
\newcommand{\VerbBar}{|}
\newcommand{\VERB}{\Verb[commandchars=\\\{\}]}
\DefineVerbatimEnvironment{Highlighting}{Verbatim}{commandchars=\\\{\}}
% Add ',fontsize=\small' for more characters per line
\newenvironment{Shaded}{}{}
\newcommand{\AlertTok}[1]{\textcolor[rgb]{1.00,0.00,0.00}{\textbf{#1}}}
\newcommand{\AnnotationTok}[1]{\textcolor[rgb]{0.38,0.63,0.69}{\textbf{\textit{#1}}}}
\newcommand{\AttributeTok}[1]{\textcolor[rgb]{0.49,0.56,0.16}{#1}}
\newcommand{\BaseNTok}[1]{\textcolor[rgb]{0.25,0.63,0.44}{#1}}
\newcommand{\BuiltInTok}[1]{\textcolor[rgb]{0.00,0.50,0.00}{#1}}
\newcommand{\CharTok}[1]{\textcolor[rgb]{0.25,0.44,0.63}{#1}}
\newcommand{\CommentTok}[1]{\textcolor[rgb]{0.38,0.63,0.69}{\textit{#1}}}
\newcommand{\CommentVarTok}[1]{\textcolor[rgb]{0.38,0.63,0.69}{\textbf{\textit{#1}}}}
\newcommand{\ConstantTok}[1]{\textcolor[rgb]{0.53,0.00,0.00}{#1}}
\newcommand{\ControlFlowTok}[1]{\textcolor[rgb]{0.00,0.44,0.13}{\textbf{#1}}}
\newcommand{\DataTypeTok}[1]{\textcolor[rgb]{0.56,0.13,0.00}{#1}}
\newcommand{\DecValTok}[1]{\textcolor[rgb]{0.25,0.63,0.44}{#1}}
\newcommand{\DocumentationTok}[1]{\textcolor[rgb]{0.73,0.13,0.13}{\textit{#1}}}
\newcommand{\ErrorTok}[1]{\textcolor[rgb]{1.00,0.00,0.00}{\textbf{#1}}}
\newcommand{\ExtensionTok}[1]{#1}
\newcommand{\FloatTok}[1]{\textcolor[rgb]{0.25,0.63,0.44}{#1}}
\newcommand{\FunctionTok}[1]{\textcolor[rgb]{0.02,0.16,0.49}{#1}}
\newcommand{\ImportTok}[1]{\textcolor[rgb]{0.00,0.50,0.00}{\textbf{#1}}}
\newcommand{\InformationTok}[1]{\textcolor[rgb]{0.38,0.63,0.69}{\textbf{\textit{#1}}}}
\newcommand{\KeywordTok}[1]{\textcolor[rgb]{0.00,0.44,0.13}{\textbf{#1}}}
\newcommand{\NormalTok}[1]{#1}
\newcommand{\OperatorTok}[1]{\textcolor[rgb]{0.40,0.40,0.40}{#1}}
\newcommand{\OtherTok}[1]{\textcolor[rgb]{0.00,0.44,0.13}{#1}}
\newcommand{\PreprocessorTok}[1]{\textcolor[rgb]{0.74,0.48,0.00}{#1}}
\newcommand{\RegionMarkerTok}[1]{#1}
\newcommand{\SpecialCharTok}[1]{\textcolor[rgb]{0.25,0.44,0.63}{#1}}
\newcommand{\SpecialStringTok}[1]{\textcolor[rgb]{0.73,0.40,0.53}{#1}}
\newcommand{\StringTok}[1]{\textcolor[rgb]{0.25,0.44,0.63}{#1}}
\newcommand{\VariableTok}[1]{\textcolor[rgb]{0.10,0.09,0.49}{#1}}
\newcommand{\VerbatimStringTok}[1]{\textcolor[rgb]{0.25,0.44,0.63}{#1}}
\newcommand{\WarningTok}[1]{\textcolor[rgb]{0.38,0.63,0.69}{\textbf{\textit{#1}}}}
\usepackage{longtable,booktabs,array}
\usepackage{calc} % for calculating minipage widths
% Correct order of tables after \paragraph or \subparagraph
\usepackage{etoolbox}
\makeatletter
\patchcmd\longtable{\par}{\if@noskipsec\mbox{}\fi\par}{}{}
\makeatother
% Allow footnotes in longtable head/foot
\IfFileExists{footnotehyper.sty}{\usepackage{footnotehyper}}{\usepackage{footnote}}
\makesavenoteenv{longtable}
\setlength{\emergencystretch}{3em} % prevent overfull lines
\providecommand{\tightlist}{%
  \setlength{\itemsep}{0pt}\setlength{\parskip}{0pt}}
\setcounter{secnumdepth}{-\maxdimen} % remove section numbering
\ifLuaTeX
  \usepackage{selnolig}  % disable illegal ligatures
\fi
\IfFileExists{bookmark.sty}{\usepackage{bookmark}}{\usepackage{hyperref}}
\IfFileExists{xurl.sty}{\usepackage{xurl}}{} % add URL line breaks if available
\urlstyle{same}
\hypersetup{
  pdftitle={File Conversion Among MD Simulation Engines Using ParmEd},
  pdfauthor={Xufan},
  hidelinks,
  pdfcreator={LaTeX via pandoc}}

\title{File Conversion Among MD Simulation Engines Using ParmEd}
\author{Xufan}
\date{}

\begin{document}
\maketitle

\href{https://parmed.github.io/ParmEd/html/index.html}{ParmEd} is a
versatile Python library that facilitates the interconversion of files
between popular molecular dynamics (MD) simulation engines like Gromacs,
Amber, and NAMD (CHARMM). This tool is especially useful for researchers
and students working in molecular dynamics who need to switch between
simulation packages without hassle. For example, you want to avoid
setting up a protein-ligand complex in Gromacs (adding ligands to gmx
force field files can be troublesome!) but do want to run MD simulations
in Gromacs for its speed. You will need to use ParmEd to convert the
Amber files to Gromacs format.

Note that the MD engine uses different algorithms and settings. You
cannot either adopt special settings in another MD engine (e.g.
restraints, you should set it up again). You should not even wish to
fully replicate a Gromacs simulation in Amber. But for most biological
systems (e.g. the solvent is not that important), MD engine usually
affects your simulation much less than other options, like the choice of
force field. So feel free to switch between MD engines!

Jump to the \protect\hyperlink{code}{code section} if you want a
solution only.

\hypertarget{installing-parmed}{%
\section{Installing ParmEd}\label{installing-parmed}}

Here's how you can
\href{https://anaconda.org/conda-forge/parmed}{install ParmEd using
Anaconda}:

\begin{Shaded}
\begin{Highlighting}[]
\ExtensionTok{conda}\NormalTok{ install }\AttributeTok{{-}c}\NormalTok{ conda{-}forge parmed}
\end{Highlighting}
\end{Shaded}

If you have compiled Amber on your system, you might already have ParmEd
installed as part of the AmberTools suite. To ensure it is properly
integrated, refer to the comprehensive guide on
\href{https://www.bilibili.com/read/cv23212288/}{compiling Amber}, which
is particularly useful if you are setting up everything from scratch.

\hypertarget{introduction}{%
\section{Introduction}\label{introduction}}

\hypertarget{knowing-the-file-formats}{%
\subsection{Knowing the file formats}\label{knowing-the-file-formats}}

These file formats are what we need in MD simulations:

\begin{longtable}[]{@{}lllll@{}}
\toprule\noalign{}
Engine & Construction Tool & Topology file & Coordinate file & Parameter
file \\
\midrule\noalign{}
\endhead
\bottomrule\noalign{}
\endlastfoot
Gromacs & \texttt{pdb2gmx} & \texttt{.top}/\texttt{.itp} & \texttt{.gro}
& -\/- \\
Amber & \texttt{tleap} & \texttt{.prmtop} & \texttt{.inpcrd} & -\/- \\
NAMD & VMD \texttt{psfgen} & \texttt{.psf} & \texttt{.pdb} &
\texttt{.prm} \\
\end{longtable}

\hypertarget{parmed-logics}{%
\subsection{ParmEd logics}\label{parmed-logics}}

ParmEd works simply: read in the topology and coordinate files, and
write out two files in the desired format.

ParmEd writes the parameters into \texttt{.inpcrd} (as it is) and
\texttt{.top} files. Always find \texttt{.prm} files when converting
both from and to NAMD.

\hypertarget{other}{%
\subsection{Other}\label{other}}

You can edit the system in ParmEd, which is out of the scope of this
post. The file parsing is very detailed so you can manipulate the system
as you like. Consult the
\href{https://parmed.github.io/ParmEd/html/index.html}{ParmEd
documentation} for more details.

\hypertarget{code}{%
\section{Code}\label{code}}

The following code shows a framework of file conversion. It implements
the basic residue renumbering function: you can set the starting residue
number. The command is

\begin{Shaded}
\begin{Highlighting}[]
\ExtensionTok{python}\NormalTok{ xxx.py }\OperatorTok{\textless{}}\NormalTok{system\_name}\OperatorTok{\textgreater{}} \OperatorTok{\textless{}}\NormalTok{starting\_residue\_number}\OperatorTok{\textgreater{}}
\end{Highlighting}
\end{Shaded}

Your topolgy and coordinate files should be named
\texttt{\textless{}system\_name\textgreater{}.xxx} both. Note that we
use \texttt{offset-1} in the code since by default ParmEd residue
numbering starts from 1.

\hypertarget{from-amber-to-gromacs}{%
\subsection{From Amber to Gromacs}\label{from-amber-to-gromacs}}

\begin{Shaded}
\begin{Highlighting}[]
\CommentTok{\# python amber2gmx\_via\_parmed.py pro 689}
\ImportTok{import}\NormalTok{ parmed }\ImportTok{as}\NormalTok{ pmd}
\ImportTok{import}\NormalTok{ sys}

\NormalTok{prefix }\OperatorTok{=}\NormalTok{ sys.argv[}\DecValTok{1}\NormalTok{]}
\NormalTok{offset }\OperatorTok{=} \BuiltInTok{int}\NormalTok{(sys.argv[}\DecValTok{2}\NormalTok{])}
\NormalTok{amber }\OperatorTok{=}\NormalTok{ pmd.load\_file(prefix}\OperatorTok{+}\StringTok{\textquotesingle{}.prmtop\textquotesingle{}}\NormalTok{, prefix}\OperatorTok{+}\StringTok{\textquotesingle{}.inpcrd\textquotesingle{}}\NormalTok{)}

\ControlFlowTok{for}\NormalTok{ residue }\KeywordTok{in}\NormalTok{ amber.residues:}
\NormalTok{    \_ }\OperatorTok{=}\NormalTok{ residue.idx  }\CommentTok{\# Get the original index}
\NormalTok{    residue.\_idx }\OperatorTok{+=}\NormalTok{ offset}\OperatorTok{{-}}\DecValTok{1}
\NormalTok{    residue.number }\OperatorTok{+=}\NormalTok{ offset}\OperatorTok{{-}}\DecValTok{1}

\CommentTok{\# Save the modified structure in Gromacs format}
\NormalTok{amber.save(prefix}\OperatorTok{+}\StringTok{\textquotesingle{}.top\textquotesingle{}}\NormalTok{, overwrite}\OperatorTok{=}\VariableTok{True}\NormalTok{, combine}\OperatorTok{=}\StringTok{\textquotesingle{}all\textquotesingle{}}\NormalTok{)}
\NormalTok{amber.save(prefix}\OperatorTok{+}\StringTok{\textquotesingle{}.gro\textquotesingle{}}\NormalTok{, overwrite}\OperatorTok{=}\VariableTok{True}\NormalTok{, combine}\OperatorTok{=}\StringTok{\textquotesingle{}all\textquotesingle{}}\NormalTok{)}
\end{Highlighting}
\end{Shaded}

Gromacs sub-topology \texttt{.itp} files can be read, but cannot be
written, i.e. ParmEd writes huge topology/coordinate files as in
Amber/NAMD. Too many water molecules may slow down the conversion.

\hypertarget{from-namd-to-gromacs}{%
\subsection{From NAMD to Gromacs}\label{from-namd-to-gromacs}}

\begin{Shaded}
\begin{Highlighting}[]
\CommentTok{\# python charmm2gmx\_via\_parmed.py pro 689}

\ImportTok{import}\NormalTok{ parmed }\ImportTok{as}\NormalTok{ pmd }
\ImportTok{from}\NormalTok{ parmed.charmm }\ImportTok{import}\NormalTok{ CharmmParameterSet}
\ImportTok{import}\NormalTok{ sys}
\NormalTok{prefix }\OperatorTok{=}\NormalTok{ sys.argv[}\DecValTok{1}\NormalTok{]}
\NormalTok{offset }\OperatorTok{=} \BuiltInTok{int}\NormalTok{(sys.argv[}\DecValTok{2}\NormalTok{])}

\NormalTok{structure }\OperatorTok{=}\NormalTok{ pmd.load\_file(prefix}\OperatorTok{+}\StringTok{\textquotesingle{}.psf\textquotesingle{}}\NormalTok{)}
\ControlFlowTok{for}\NormalTok{ residue }\KeywordTok{in}\NormalTok{ structure.residues:}
\NormalTok{    \_ }\OperatorTok{=}\NormalTok{ residue.idx}
\NormalTok{    residue.\_idx }\OperatorTok{+=}\NormalTok{ offset}\OperatorTok{{-}}\DecValTok{1}
\NormalTok{    residue.number }\OperatorTok{+=}\NormalTok{ offset}\OperatorTok{{-}}\DecValTok{1}
\NormalTok{parameter }\OperatorTok{=}\NormalTok{ CharmmParameterSet(}\StringTok{\textquotesingle{}par\_all36m\_prot.prm\textquotesingle{}}\NormalTok{, }\StringTok{\textquotesingle{}toppar\_water\_ions\_namd.str\textquotesingle{}}\NormalTok{)  }\CommentTok{\# add more if necessary}

\CommentTok{\# parmed does not realize that gmx adopts the absolute value while charmm files store the real value (negative!)}
\ControlFlowTok{for}\NormalTok{ atomname, atomtype }\KeywordTok{in}\NormalTok{ parameter.atom\_types.items():}
\NormalTok{    atomtype.epsilon }\OperatorTok{*=} \OperatorTok{{-}}\DecValTok{1}
\NormalTok{    atomtype.epsilon\_14 }\OperatorTok{*=} \OperatorTok{{-}}\DecValTok{1}
\NormalTok{structure.load\_parameters(parameter)}

\CommentTok{\# Save the modified structure in Gromacs format}
\NormalTok{structure.save(prefix}\OperatorTok{+}\StringTok{\textquotesingle{}.top\textquotesingle{}}\NormalTok{, overwrite}\OperatorTok{=}\VariableTok{True}\NormalTok{, combine}\OperatorTok{=}\StringTok{\textquotesingle{}all\textquotesingle{}}\NormalTok{)}
\NormalTok{structure }\OperatorTok{=}\NormalTok{ pmd.load\_file(prefix}\OperatorTok{+}\StringTok{\textquotesingle{}.pdb\textquotesingle{}}\NormalTok{)}
\NormalTok{structure.save(prefix}\OperatorTok{+}\StringTok{\textquotesingle{}.gro\textquotesingle{}}\NormalTok{, overwrite}\OperatorTok{=}\VariableTok{True}\NormalTok{, combine}\OperatorTok{=}\StringTok{\textquotesingle{}all\textquotesingle{}}\NormalTok{)}
\end{Highlighting}
\end{Shaded}

\textbf{Note}\\

In parameter files like \texttt{par\_all36m\_prot.prm} downloaded from
\href{http://mackerell.umaryland.edu/charmm_ff.shtml}{CHARMM website},
officially all atom type definitions are commented, but we should
uncomment them for parmed, or it cannot find atomtypes. Double check
your files!

\hypertarget{from-gromacs-to-amber}{%
\subsection{From Gromacs to Amber}\label{from-gromacs-to-amber}}

\begin{Shaded}
\begin{Highlighting}[]
\CommentTok{\# python gmx2amber.py system}

\ImportTok{import}\NormalTok{ parmed }\ImportTok{as}\NormalTok{ pmd}
\ImportTok{import}\NormalTok{ sys}
\NormalTok{prefix }\OperatorTok{=}\NormalTok{ sys.argv[}\DecValTok{1}\NormalTok{]}

\NormalTok{parm }\OperatorTok{=}\NormalTok{ pmd.load\_file(prefix}\OperatorTok{+}\StringTok{\textquotesingle{}.top\textquotesingle{}}\NormalTok{, prefix}\OperatorTok{+}\StringTok{\textquotesingle{}.gro\textquotesingle{}}\NormalTok{)}
\NormalTok{parm.write(prefix}\OperatorTok{+}\StringTok{\textquotesingle{}.prmtop\textquotesingle{}}\NormalTok{)}
\NormalTok{parm.write(prefix}\OperatorTok{+}\StringTok{\textquotesingle{}.inpcrd\textquotesingle{}}\NormalTok{)}
\end{Highlighting}
\end{Shaded}

I actually have not tried this (see
\protect\hyperlink{problems}{problems}). You may need to add residue
renumbering mechanisms. Practice yourself! And I guess From CHARMM to
Gromacs works similarly.

\hypertarget{renumber-gmx-files}{%
\subsection{Renumber gmx files}\label{renumber-gmx-files}}

This adopts the similar process.

\begin{Shaded}
\begin{Highlighting}[]
\CommentTok{\# python gmx\_renumber\_via\_parmed.py pro 689}

\ImportTok{import}\NormalTok{ parmed }\ImportTok{as}\NormalTok{ pmd }
\ImportTok{import}\NormalTok{ sys}
\NormalTok{prefix }\OperatorTok{=}\NormalTok{ sys.argv[}\DecValTok{1}\NormalTok{]}
\NormalTok{offset }\OperatorTok{=} \BuiltInTok{int}\NormalTok{(sys.argv[}\DecValTok{2}\NormalTok{])}
\NormalTok{gmx }\OperatorTok{=}\NormalTok{ pmd.load\_file(prefix}\OperatorTok{+}\StringTok{\textquotesingle{}.top\textquotesingle{}}\NormalTok{, prefix}\OperatorTok{+}\StringTok{\textquotesingle{}.gro\textquotesingle{}}\NormalTok{)}

\ControlFlowTok{for}\NormalTok{ residue }\KeywordTok{in}\NormalTok{ gmx.residues:}
\NormalTok{    \_ }\OperatorTok{=}\NormalTok{ residue.idx}
\NormalTok{    residue.\_idx }\OperatorTok{+=}\NormalTok{ offset}\OperatorTok{{-}}\DecValTok{1}
\NormalTok{    residue.number }\OperatorTok{+=}\NormalTok{ offset}\OperatorTok{{-}}\DecValTok{1}
\NormalTok{gmx.remake\_parm()}
\NormalTok{gmx.save(prefix}\OperatorTok{+}\StringTok{\textquotesingle{}.top\textquotesingle{}}\NormalTok{, overwrite}\OperatorTok{=}\VariableTok{True}\NormalTok{)}
\NormalTok{gmx.save(prefix}\OperatorTok{+}\StringTok{\textquotesingle{}.gro\textquotesingle{}}\NormalTok{, overwrite}\OperatorTok{=}\VariableTok{True}\NormalTok{)}
\end{Highlighting}
\end{Shaded}

Suppose you have the following CHARMM files:

\begin{itemize}
\item
  Topology file (RTF): top\_all36\_prot.rtf
\item
  Parameter file (PAR): par\_all36\_prot.prm
\item
  Protein Structure File (PSF): structure.psf
\end{itemize}

To convert these files to Amber format:

\begin{Shaded}
\begin{Highlighting}[]
\ExtensionTok{chamber} \AttributeTok{{-}top}\NormalTok{ topol.rtf }\AttributeTok{{-}param}\NormalTok{ params.par }\AttributeTok{{-}psf}\NormalTok{ structure.psf }\AttributeTok{{-}crd}\NormalTok{ structure.crd }\AttributeTok{{-}outparm}\NormalTok{ amber.prmtop }\AttributeTok{{-}outcrd}\NormalTok{ amber.inpcrd}
\end{Highlighting}
\end{Shaded}

\hypertarget{residue-renumbering}{%
\section{Residue renumbering}\label{residue-renumbering}}

\hypertarget{problem}{%
\subsection{Problem}\label{problem}}

None of these file formats are perfect.

\begin{itemize}
\item
  Gromacs files do not have chain identifiers. By default chains are
  separated into a few \texttt{.itp} files, so it\textquotesingle s hard
  to locate an atom in a specific chain in a \texttt{.gro} file.
\item
  Amber files always start with residue numbers 1, which causes trouble
  when aligning with the "biological" residue nubmers.
\item
  VMD files have full identifiers. However, we have to manually separate
  the chains when modeling.
\end{itemize}

You cannot change the file formats unless your write your own MD engine.
So just put up with it...

With ParmEd, you can try to edit the residue numbers to match the
"biological" residue numbers. Sadly, if you have multiple chains and
they are overlapping, you still have to

\hypertarget{edit-in-parmed}{%
\subsection{Edit in ParmEd}\label{edit-in-parmed}}

In ParmEd, every \texttt{Residue} object in a \texttt{Structure} has an
\texttt{idx} attribute. This attribute indicates the
residue\textquotesingle s index within the structure, and it is managed
internally by ParmEd. It is crucial not to modify this attribute
directly, as it could lead to inconsistent state within the structure.

While the \texttt{idx} attribute is protected, you can adjust residue
numbers for Amber files by using an offset. This adjustment is
particularly useful when preparing files for simulation in different MD
engines.

\hypertarget{parameters-and-atomtypes}{%
\section{Parameters and atomtypes}\label{parameters-and-atomtypes}}

\hypertarget{gromacs-independent-parameter-specification}{%
\subsection{GROMACS: Independent Parameter
Specification}\label{gromacs-independent-parameter-specification}}

In GROMACS, topology files (typically \texttt{.top}) allow for each bond
term to be specified independently. This means that different bond
parameters can be assigned to the same pair of atom types, provided they
occur in different contexts within the molecule. This flexible approach
accommodates complex molecules with asymmetrical properties, such as
those often created using quantum mechanical (QM) optimizations.

Example of a GROMACS bond specification:

\begin{Shaded}
\begin{Highlighting}[]
\NormalTok{; Bond parameters}
\NormalTok{; i    j   func   length  force\_const}
\NormalTok{  1    2    1      0.123   456.7   ; Asymmetric bond A}
\NormalTok{  2    3    1      0.123   456.7   ; Asymmetric bond B}
\end{Highlighting}
\end{Shaded}

\hypertarget{charmm-type-based-parameter-definition}{%
\subsection{CHARMM: Type-Based Parameter
Definition}\label{charmm-type-based-parameter-definition}}

Conversely, CHARMM typically defines parameters between different atom
types based on a consistent set of parameters across all bonds involving
those atom types. This approach assumes that identical pairs of atom
types will always exhibit the same bonding characteristics, regardless
of their molecular environment.

\begin{Shaded}
\begin{Highlighting}[]
\NormalTok{BONDS}
\NormalTok{CA   CB   340.0  1.529   ; Standard peptide bond}
\NormalTok{CA   CG   317.0  1.510   ; Standard alkane bond}
\end{Highlighting}
\end{Shaded}

\hypertarget{resolving-parameter-inconsistencies}{%
\subsection{Resolving Parameter
Inconsistencies}\label{resolving-parameter-inconsistencies}}

When converting from GROMACS to CHARMM formats using tools like ParmEd,
discrepancies in how bond parameters are specified can lead to errors.
For instance, ParmEd might encounter a \texttt{ParameterError} if it
detects different bond parameters for the same atom types, which is
permissible in GROMACS but not in CHARMM. This issue is particularly
evident with complex ions or molecules optimized asymmetrically through
QM methods, such as \(\ce{Al(OH)(H2O)5^2+}\).

To address these conversion challenges, users have two main options:

\begin{enumerate}
\def\labelenumi{\arabic{enumi}.}
\item
  \textbf{Assign Different Atom Types:} Modify the topology to assign
  unique atom types for bonds that require different parameters.
\item
  \textbf{Uniform Bond Parameters:} Standardize bond parameters for each
  pair of atom types, ensuring consistency across the entire molecule.
\end{enumerate}

For more details on handling these conversions and the underlying code
structure of ParmEd, consider exploring the following resources:

\begin{itemize}
\item
  \href{https://github.com/ParmEd/ParmEd}{ParmEd GitHub repository}
\item
  \href{https://github.com/ParmEd/ParmEd/issues/1111}{Issue related to
  parameter mismatches}
\item
  \href{https://github.com/ParmEd/ParmEd/issues/968}{Discussion on
  handling different parameters}
\end{itemize}

\hypertarget{end}{%
\section{End}\label{end}}

We welcome your feedback and contributions! If you have developed new
workflows or if you encounter any issues, please don\textquotesingle t
hesitate to reach out. For reporting problems, consider opening an issue
on the \href{https://github.com/ParmEd/ParmEd/issues}{ParmEd GitHub
repository}. Your insights and experiences are invaluable in enhancing
the tools and community resources.

\end{document}
